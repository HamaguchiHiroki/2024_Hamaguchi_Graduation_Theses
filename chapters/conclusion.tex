\newpage
\section{結言}
本研究では手作業で作製可能な外径3 mmの細径MPAの作製方法を開発し,それを用いた羽状筋の構成手法の開発,ズワイガニの歩脚をモデルにしたロボットの開発を行った.
まず,4つの節に分かれた2次元3自由度の機体を開発しそれらを用いて細径MPAを羽状配置にした際に動作するか,また,どのような問題点があるか予備実験によって確認した.
次に,ズワイガニを解剖して得られたデータを基にズワイガニの歩脚を模したロボットを作製し予備実験で見つかった課題を解決できるように細径MPAの改善,新たな集積方法の開発を行い動作実験を行った.
動作実験では長節から腕節間,腕節から前節間の開閉動作を確認することが出来た.

今後は,より蟹らしい歩脚の動作を実現するために節間膜の実装や羽状筋の配置の見直しを行い,前節から指節間にも集積したMPAを配置し歩脚ロボットの完成を目指す.