\newpage
\section{結言}
本研究では,羽状筋の3つの課題について対策を講じ,改良型細径空圧羽状筋の開発を行うことができた.
改良したことにより,細径MPAの作製が容易かつ収縮率が向上し,羽状角を変化することが確認された.
そして,外骨格を設計する段階で数理モデルを用いて,各関節における可動域の計算を行った.
その後,改良型細径空圧羽状筋と外骨格を用いて歩脚ロボットを作製し,動作実験を行った.
動作実験では長節から腕節間,腕節から前節間の開閉動作を確認することが出来た.
その結果,全ての関節において多少の誤差はあるものの,概ね設計通りの可動域が実現されていることが確認できた.


今後は多様な長さの細径MPAを配置する方法,細径MPAに圧縮空気を印加していない状態で張力が発生しない固定方法を考えることにより実際の蟹の動きに近づけるような歩脚ロボットの開発を予定している.