\newpage
\section*{Abstract}
The McKibben Pneumatic Actuator (MPA) is an actuator that contracts when compressed air is applied to it, generating tension in its own axial direction.
Recently, thin MPAs with a diameter of a few millimeters have been attracting attention.
On the other hand, biomimetic robots with exoskeletons that mimic crustaceans and insects have mainly used wire drives and servo motors due to the difficulty of placing actuators inside the exoskeleton, but there are some differences from actual biological structures.
In response to this problem, previous research developed a feather-shaped muscle using a thin MPA. In addition, a gait robot was developed based on measurement data obtained by dissecting a snow crab, and the joints were successfully opened and closed using thin pneumatic pterostome muscles.
On the other hand, there are still various issues to be solved, such as the production method of the thin MPA, the composition of the pterygoid muscles, and the reproduction of the joint structure.
In this study, we developed a robot with a thin pneumatic winged muscle that improves on these issues in previous studies.
We confirmed that all joints were able to achieve the designed range of motion.
In particular, we confirmed that the range of motion of the joint between the long segment and the brachial segment was equivalent to that of an actual crab.
In the future, we will improve the method of tendon fixation and the mathematical model used to calculate the range of motion, and aim to develop a walking leg robot that can move in a range of motion similar to that of an actual crab.