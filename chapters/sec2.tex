\newpage
\section{先行研究の内容}
%%%%%%%%%%%%%%%%%%%%%%%%%%%%%%%%%%%%%%%%%%%%%%%%%%%%%%%%%
\subsection{McKibben型空気圧人工筋肉アクチュエータ(MPA)}
% ・今のとこは先輩のやつを真似ただけ.
MPAはシリコンゴムチューブをナイロンメッシュで覆うことで構成されており(図\ref{fig:MPA}\subref{fig:Structure}),両端に栓をするシンプルな構造である.
これに圧縮した空気を印加することでシリコンゴムチューブが膨張しメッシュによる自身の軸方向への張力が発生するアクチュエータである(図\ref{fig:MPA}\subref{fig:move}).
高出力かつ素材自体も軽量で,物理的柔軟性による高い弾性力を持つという利点があり,筋肉の代用として生物を模したロボットやリハビリなどに用いられる.
しかし図\ref{fig:MPA}に示すような従来の直径数十 mmのMPAは膨張時の径の拡大が大きいため配置の際は膨張を阻害しないような配置や直線形状で駆動するような取り付け方が求められ,
取り付けの位置や密度に制限がある.
%%%%%%%%%%%%%%%%%%%%%%%%%%%%%%%%%%%%%%%%%%%%%%%%%%%%%%%%%
\begin{figure}[b]
  %
  \begin{minipage}{0.49\columnwidth}
    \vspace{4mm}
    \centering
    \includegraphics[scale=1]{image/MPA_kousei.png}
    \vspace{3mm}
    \subcaption{MPA断面図}
    \label{fig:Structure}
  \end{minipage}
  %
  \begin{minipage}{0.49\columnwidth}
    \vspace{25mm}
    \centering
    \includegraphics[scale=.8]{image/MPA_dousa.png}
    \subcaption{MPA外観および動作の様子}
    \label{fig:move}
  \end{minipage}
  %
  \caption{McKibben型空気圧人工筋(MPA)の構成および外観\cite{中西大輔2020}}
  \label{fig:MPA}
\end{figure}
%%%%%%%%%%%%%%%%%%%%%%%%%%%%%%%%%%%%%%%%%%%%%%%%%%%%%%%%%
\subsection{細径MPA}
% ・脇本修一の論文を参考に細径MPAの冗長性,従来のMPAとの比較などについて説明,φ3のMPAの写真
本研究で用いる細径のMPAについて説明する.図\ref{fig:campare}に本研究で開発に成功した外径5 mmと3 mmの細径MPAと外径12 mmの従来のMPAを示す.
細径化には下記のような利点があると考えられている\cite{wakimoto}\cite{1390282680917523328}.
\begin{enumerate}
  \item 非常にしなやかな人工筋となり座屈することなく任意形状での配置や集積が可能
  \item 集積化により収縮量を増大させることが可能
  \item 集積化により冗長性を持ちシステムの安全性が向上
\end{enumerate}
MPAの収縮力は断面積に比例するため,細径MPAは従来のものに比べると発生する張力は小さいものの,細くしなやかであり任意形状での配置や集積化が可能である.
生体筋と柔らかさや動作が似ていることから,紡錘状に集積し筋骨格系ロボット(図\ref{fig:saikei}\subref{fig:kin})へ応用したり,生物模倣ロボットとしてタコ腕模倣メカニズム(図\ref{fig:saikei}\subref{fig:tako})も開発されており,曲げ動作やねじり動作を実現している\cite{森和也2014}.
%%%%画像置き場%%%%%%%%%%%%%%%%%%%%%%%%%%%%%%%%%%%%%%%%%%%%
\begin{figure}[t]
  \centering
  \includegraphics[scale=0.7]{image/hikaku.jpg}
  \caption{MPAの外径の比較(左から12,5,3 mm)}
  \label{fig:campare}
\end{figure}
%
\begin{figure}
  %
  \begin{minipage}{0.5\columnwidth}
    \centering  
    \includegraphics[scale=0.5]{image/kinkokkaku.JPG}
    \subcaption{筋骨格系ロボット\cite{森田隆介2016}}
    \label{fig:kin}
  \end{minipage}
  %
  \begin{minipage}{0.5\columnwidth}
    \centering
    \vspace{5mm}
    \includegraphics[scale=0.57]{image/takoJPG.JPG}
    \subcaption{タコ腕模倣メカニズム\cite{森和也2014}}
    \label{fig:tako}
  \end{minipage}
  %
  \caption{細径空圧筋を用いたロボット}
  \label{fig:saikei}
\end{figure}
  %
\begin{figure}[t]
  %
  \begin{minipage}{1\columnwidth}
    \centering
    \includegraphics[scale=0.1]{image/syusyuku.png}
    \vspace{-3mm}
    \caption{収縮率}
    \label{fig:syusyuku}
  \end{minipage}
  %
  \begin{minipage}{1\columnwidth}
    \centering
    \vspace{3mm}
    \includegraphics[scale=0.3]{image/method.png}
    \caption{細径MPAの作製手順}
    \label{fig:method}
  \end{minipage}
  %
\end{figure}
%
%%%%%%%%%%%%%%%%%%%%%%%%%%%%%%%%%%%%%%%%%%%%%%%%%%%%%%%%%
\subsection{先行研究の成果と課題}



