\newpage
\setcounter{page}{1}
\section{緒言}
McKibben型空気圧人工筋(McKibben Pneumatic Actuator,以下MPA)は圧縮空気を入力することで収縮し,自身の軸方向への張力を発生させるアクチュエータである\cite{2003722}.
従来は直径が数十mm程度のMPAを用いたロボットに関する応用研究が盛んに行われてきたが,近年では直径が数mm程度の細径MPAが注目を集めている\cite{wakimoto}.
細径MPAは従来のものより細くしなやかであり生体筋に似た特徴から小さな筋肉,あるいは集積によって単純な紡錘型以外の筋肉を表現可能なため,筋骨格系ロボットや生物模倣ロボットなどに盛んに用いられてきた\cite{森田隆介2016}\cite{森和也2014}.

一方で甲殻類や昆虫などを模した外骨格生物模倣ロボットに関しては外骨格内部へアクチュエータを配置することが困難なことから,ワイヤ駆動や関節にサーボモータを配置したもの\cite{jmse10121804}が主流となっている.
このようなロボットは外骨格生物の外形こそ再現できているものの,実際の生物の構成や駆動原理からして異なる.
また「構成要素が外骨格内にすべて納まっている」という外骨格ならではのメリットも,実現できているとは言い難い.
これに対して前述の細径MPAは,その細さやしなやかさから細長い外骨格内部に配置可能であり,また集積することで実際の生物のような羽状筋も表現することが可能である\cite{2003}.

そこで本研究では細径MPAを用いた外骨格生物模倣ロボットの開発を提案する.
本稿では初期段階として,外骨格生物のなかでも甲殻類の蟹をモデル生物として設定し,これの歩脚を模倣したロボットの開発に取り組む.
まず先行研究\cite{hasegawa}で作製された歩脚ロボットにおける課題点を解説する.
先行研究で用いられた細径MPA作製方法は煩雑かつ時間と練度が必要,細径MPAの収縮性能が低いことと羽状筋の細径MPAの根元の角度が固定されていて腱の引き込みを妨げている課題が見られた.
そこで本研究では,先行研究で行われた実際の蟹の解剖によって得られた知見と寸法,および前述した課題点などをもとに改良した外骨格と細径MPAを用いた羽状筋を開発する.
その後,改良した外骨格と羽状筋を用いて作製した歩脚ロボットの動作実験を行う.

本論文の構成は以下の通りである.まず2章では,本研究で用いる細径MPAに関する特徴と先行研究について述べてから,本研究で開発に成功した細径MPAの作製方法を紹介する.
次に3章では,本研究でモデル生物として扱う蟹の構成と筋構造について実際にズワイガニを解剖して得た知見などを基に述べる.
最後に4章では模倣ロボットを作製するにあたって集積細径MPAを用いた羽状筋の開発について述べたのち,ズワイガニをモデルにした歩脚ロボットの開発とその動作実験について述べる.

% 代表的な人工筋肉として,圧縮空気により骨格筋のように収縮するMcKibben 型空気圧人工筋肉(MPA) があげられる.
% 従来は直径が数十mm 程度のものが多かったが,近年では数mm 程度の細径のMPA が注目を集めている\cite{wakimoto}.
% その細さを生かして小さな筋肉,あるいは集積によって単純な紡錘型以外の筋肉を表現可能なことから,筋骨格系ロボットにおいて特に盛んに用いられている\cite{wakimoto}.
% 一方で甲殻類をはじめとする外骨格系ロボットについては,ワイヤ駆動や関節にサーボモータを配置したものが主流であった\cite{jmse10121804}.
% これは骨格内部にアクチュエータを配置することが困難だからである.
% 細径MPA であれば骨格内部にアクチュエータを配置することが可能であり,実際の生物に近い構成でロボットを作製することが可能である.
% そこで本研究では外骨格生物のうち甲殻類の蟹をモデルに,実際の蟹の筋構造を参考にして細径MPA を使用した蟹の歩脚ロボットの開発に取り組む.
