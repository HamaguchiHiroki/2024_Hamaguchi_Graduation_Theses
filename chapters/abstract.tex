\newpage
\section*{概要}
McKibben型空気圧人工筋(McKibben Pneumatic Actuator,以下MPA)は圧縮空気を入力することで収縮し,自身の軸方向への張力を発生させるアクチュエータであり,近年では直径数mmの細径MPAが注目されている.
細径MPAは生体筋に似た特性を持ち,筋骨格系ロボットや生物模倣ロボットに用いられてきた.
一方,甲殻類や昆虫を模した外骨格を有する生物模倣ロボットでは,外骨格内部へのアクチュエータの配置が困難なため,主にワイヤ駆動やサーボモータを使用したものが主流であったが,実際の生物の構成と相違点が出てきてしまう問題点があった.
そこで,先行研究では細径MPAを用いた羽状筋の開発が行われた.またズワイガニを実際に解剖して得られた計測データを参考に歩脚ロボットを開発し,細径空圧羽状筋によって関節を開閉動作させることに成功した\cite{hasegawa}.
一方で,先行研究で開発された細径空圧羽状筋およびロボットにはいくつかの問題点が確認された.
本研究では先行研究で見つかった課題点を解決した改良型細径空圧羽状筋とロボットの開発を行い,作製した歩脚ロボットで動作実験を行い長節から指節までの節の開閉動作を確認した.
\newpage
\section*{Abstract}
The McKibben Pneumatic Actuator (MPA) is an actuator that contracts when compressed air is applied to it, generating tension in its own axial direction.
Recently, thin MPAs with a diameter of a few millimeters have been attracting attention.
On the other hand, biomimetic robots with exoskeletons that mimic crustaceans and insects have mainly used wire drives and servo motors due to the difficulty of placing actuators inside the exoskeleton, but there are some differences from actual biological structures.
Therefore, in a previous study, a feather-shaped muscle was developed using a thin MPA. In addition, a gait robot was developed based on measurement data obtained by actually dissecting a snow crab, and the joints were successfully opened and closed using thin pneumatic pterostomes.
On the other hand, some problems were identified in the pneumatic pterygoid muscles and the robot developed in the previous study.
In this study, we developed an improved pneumatic pterygoid muscle and robot that solved the problems found in the previous study, and confirmed the opening and closing of joints from the long segment to the phalanx using a walking leg robot we fabricated.