\newpage
\section*{概要}
McKibben型空気圧人工筋(McKibben Pneumatic Actuator,以下MPA)は圧縮空気を入力することで収縮し,自身の軸方向への張力を発生させるアクチュエータであり,近年では直径数mmの細径MPAが注目されている.
細径MPAは生体筋に似た特性を持ち,筋骨格系ロボットや生物模倣ロボットに用いられてきた.
一方,甲殻類や昆虫を模した外骨格を有する生物模倣ロボットでは,外骨格内部へのアクチュエータの配置が困難なため,主にワイヤ駆動やサーボモータを使用したものが主流であったが,実際の生物の構成と相違点が出てきてしまう問題点があった.
これに対して先行研究では細径MPAを用いた羽状筋の開発が行われた.またズワイガニを実際に解剖して得られた計測データを参考にし歩脚ロボットを開発し,細径空圧羽状筋によって関節を開閉動作させることに成功した.
一方で,細径MPAの制作方法や羽状筋の構成,および関節構造の再現など様々な点で課題が残された.
本研究ではこれらの先行研究の課題点を改良した細径空圧羽状筋とロボットの開発を行った.
長節から指節までの節の開閉動作を実機によって確認し,全ての関節について概ね設計通りの可動域が実現できていることを確認した.
特に長節-腕節間関節については,実際のカニと同等の可動域が再現できていることを確認した.
\newpage
\section*{Abstract}
The McKibben Pneumatic Actuator (MPA) is an actuator that contracts when compressed air is applied to it, generating tension in its own axial direction.
Recently, thin MPAs with a diameter of a few millimeters have been attracting attention.
On the other hand, biomimetic robots with exoskeletons that mimic crustaceans and insects have mainly used wire drives and servo motors due to the difficulty of placing actuators inside the exoskeleton, but there are some differences from actual biological structures.
In response to this problem, previous research developed a feather-shaped muscle using a thin MPA. In addition, a gait robot was developed based on measurement data obtained by dissecting a snow crab, and the joints were successfully opened and closed using thin pneumatic pterostome muscles.
On the other hand, there are still various issues to be solved, such as the production method of the thin MPA, the composition of the pterygoid muscles, and the reproduction of the joint structure.
In this study, we developed a robot with a thin pneumatic winged muscle that improves on these issues in previous studies.
We confirmed that all joints were able to achieve the designed range of motion.
In particular, we confirmed that the range of motion of the joint between the long segment and the brachial segment was equivalent to that of an actual crab.