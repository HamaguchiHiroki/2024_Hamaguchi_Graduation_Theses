\newpage
\section*{概要}
McKibben型空気圧人工筋(McKibben Pneumatic Actuator,以下MPA)は圧縮空気を入力することで収縮し,自身の軸方向への張力を発生させるアクチュエータであり,近年では直径数mmの細径MPAが注目されている.
細径MPAは生体筋に似た特性を持ち,筋骨格系ロボットや生物模倣ロボットに用いられてきた.
一方,甲殻類や昆虫を模した外骨格を有する生物模倣ロボットでは,外骨格内部へのアクチュエータの配置が困難なため,主にワイヤ駆動やサーボモータを使用したものが主流であったが,実際の生物の構成と相違点が出てきてしまう問題点があった.
本研究では細径MPAを用いた外骨格生物模倣ロボットの開発を提案し,蟹をモデルにした歩脚ロボットの開発に取り組む.
それに向けて本論文では蟹の構成や筋構造について実際にズワイガニを解剖し得られた知見を基に,細径MPAを用いた羽状筋および歩脚ロボットの開発をし動作実験を行い長節から指節までの節の開閉動作を確認した.
\newpage
\section*{Abstract}
The McKibben Pneumatic Actuator (MPA) is an actuator that contracts when compressed air is applied to it, generating tension in its own axial direction.
Recently, thin MPAs with a diameter of a few millimeters have been attracting attention.
On the other hand, biomimetic robots with exoskeletons that mimic crustaceans and insects have mainly used wire drives and servo motors due to the difficulty of placing actuators inside the exoskeleton, but there are some differences from actual biological structures.
In this study, we propose the development of an exoskeleton bio-mimetic robot using a thin MPA, and work on the development of a walking leg robot modeled after a crab.
Based on the knowledge obtained from the dissection of a snow crab, we developed a gait robot using thin MPA and its winged muscles, and conducted movement experiments to confirm the opening and closing of the joints from the long segment to the phalanges.