\newpage
\section*{概要}
McKibben型空気圧人工筋(McKibben Pneumatic Actuator,以下MPA)は圧縮空気を入力することで収縮し,自身の軸方向への張力を発生させるアクチュエータであり,近年では直径数mmの細径MPAが注目されている.
細径MPAは生体筋に似た特性を持ち,筋骨格系ロボットや生物模倣ロボットに用いられてきた.
一方,甲殻類や昆虫を模した外骨格を有する生物模倣ロボットでは,外骨格内部へのアクチュエータの配置が困難なため,主にワイヤ駆動やサーボモータを使用したものが主流であったが,実際の生物の構成と相違点が出てきてしまう問題点があった.
これに対して先行研究では細径MPAを用いた羽状筋の開発が行われた.またズワイガニを実際に解剖して得られた計測データを参考にし歩脚ロボットを開発し,細径空圧羽状筋によって関節を開閉動作させることに成功した.
一方で,細径MPAの制作方法や羽状筋の構成,および関節構造の再現など様々な点で課題が残された.
本研究ではこれらの先行研究の課題点を改良した細径空圧羽状筋とロボットの開発を行った.
長節から指節までの節の開閉動作を実機によって確認し,全ての関節について概ね設計通りの可動域が実現できていることを確認した.
特に長節-腕節間関節については,実際のカニと同等の可動域が再現できていることを確認した.
今後は腱の固定方法と可動域の計算の際に用いた数理モデルの改良を行い,実際のカニの可動域に近い動きをする歩脚ロボットの開発を目指す.
